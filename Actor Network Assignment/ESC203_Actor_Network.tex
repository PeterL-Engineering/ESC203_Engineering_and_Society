\documentclass{article} % For LaTex2e
\usepackage{iclr2022_conference,times}
% Optional math commands from https://github.com/goodfeli/dlbook_notation.
\input{math_commands.tex}

%######## ESC203: Uncomment your submission name
%\newcommand{\escname}{ - Project Proposal}
%\newcommand{\escname}{Progress Report}
%\newcommand{\escname}{Final Report}

%######## ESC203: Put your Group Number here
%\newcommand{\gpnumber}{40}

\usepackage{hyperref}
\usepackage{xcolor}
\usepackage[normalem]{ulem}
\usepackage{url}
\usepackage{graphicx}
\usepackage{placeins}
\usepackage{float}
\usepackage{tikz}
\usepackage{multicol}

%######## ESC203: Put your project Title here
\title{Discovering Musical Motivations via \\
Actor Network Theory}

%######## ESC203: Put your names, student IDs and Emails here
\author{Peter Leong \\
Student\# 1010892955 \\
peter.leong@mail.utoronto.ca \\
\AND
}

% The \author macro works with any number of authors. There are two commands
% used to separate the names and addresses of multiple authors: \And and \AND.
%
% Using \And between authors leaves it to \LaTex{} to determine where to break
% the lines. Using \AND forces a linebreak at that point. So, if \LaTex{}
% puts 3 of 4 authors names on the first line, and the last on the second
% line, try using \AND instead of \And before the third author name.

\newcommand{\fix}{\marginpar{FIx}}
\newcommand{\new}{\marginpar{NEW}}

\iclrfinalcopy 
%######## ESC203: Document starts here
\begin{document}

\maketitle

\vspace{-6ex}

\begin{abstract}
%######## ESC203: Do not change the next line. This shows your Main body page count.
----Total Pages: \pageref{last_page}
\end{abstract}

\vspace{2ex}

\begin{multicols}{2}

\section{Introduction}

Testing for change

\section{Maintaing Order Through Tradition}

Entering a musical space brings about friendships and connections that can last a lifetime.
Having been at an instituition that fostered these connections for a decade of my upbringing, many of the relationships I have formed have been rooted in musical experiences.
One such tradition that has been a staple of the choir school is \textit{Tenebrae}.

\textit{Tenebrae} has been one of if not the longest standing choral tradition at St. Michael's Choir School that brings together generations of choir boys.
Each year during the Wednesday of Holy Week, current students and alumni perform one of the most well-known works of the school's founder, Monsignor John Edward Ronan.
Though during my years as a student at the choir school, \textit{Tenebrae} was already one of my favourite services as it brought together older students who I met at the school, it has become even more so now that I myself am an alumnus.

\subsection{Staying Connected Through Music}

One of the greatest beauties of \textit{Tenebrae} is that it would not be possible without the participation of alumni.
As a result of the consistency that has endured across decades of \textit{Tenebrae} performances, various actors within the network such as the cathedral basilica and its rectors, the choir school, and the alumni association have become locked into the roles they take on.
Additionally, due to the minimal changes in each actors' roles, the \textit{Tenebrae} service has largely become punctualized into a single actor whose smaller influencing actants need not be analysed at the microscopic scale.

It is not surprising that over the years, there have been a wide variety of musical styles and interpretations of the \textit{Tenebrae} responsories (the music sung during the service); however, despite this resistance in the network, its effects have largely been hidden.


\section{Forming New Connections}

\subsection{Post-Secondary Choral Involvement}



\section{Conclusion}



%\bibliographystyle{iclr2022_conference}
%\bibliography{ESC203_Actor_Network}

\end{multicols}

\label{last_page}

\end{document}