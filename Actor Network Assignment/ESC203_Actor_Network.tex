\documentclass{article} % For LaTex2e
\usepackage{iclr2022_conference,times}
% Optional math commands from https://github.com/goodfeli/dlbook_notation.
\input{math_commands.tex}

%######## ESC203: Uncomment your submission name
%\newcommand{\escname}{ - Project Proposal}
%\newcommand{\escname}{Progress Report}
%\newcommand{\escname}{Final Report}

%######## ESC203: Put your Group Number here
%\newcommand{\gpnumber}{40}

\usepackage{hyperref}
\usepackage{xcolor}
\usepackage[normalem]{ulem}
\usepackage{url}
\usepackage{graphicx}
\usepackage{placeins}
\usepackage{float}
\usepackage{tikz}
\usepackage{multicol}

%######## ESC203: Put your project Title here
\title{An Actor-Network Analysis of Musical Development:\\
Tradition, Pedagogy, and the Pursuit of Beauty}

%######## ESC203: Put your names, student IDs and Emails here
\author{Peter Leong \\
Student\# 1010892955 \\
peter.leong@mail.utoronto.ca \\
\AND
}

% The \author macro works with any number of authors. There are two commands
% used to separate the names and addresses of multiple authors: \And and \AND.
%
% Using \And between authors leaves it to \LaTex{} to determine where to break
% the lines. Using \AND forces a linebreak at that point. So, if \LaTex{}
% puts 3 of 4 authors names on the first line, and the last on the second
% line, try using \AND instead of \And before the third author name.

\newcommand{\fix}{\marginpar{FIx}}
\newcommand{\new}{\marginpar{NEW}}

\iclrfinalcopy 
%######## ESC203: Document starts here
\begin{document}

\vspace{-4ex}

\maketitle

\vspace{-12ex}

\begin{abstract}
This reflection examines my musical journey through the lens of Actor-Network Theory (ANT), exploring how human and non-human actors form networks that create meaning and motivation. 
I analyze the stability of the longstanding Tenebrae tradition at my choir school, which punctualizes a complex network of alumni, institutions, and rituals into a single enduring actant. 
In stark contrast, the profound loss of my piano teacher demonstrates how the dissolution of a central actor forces network reconfiguration and transforms active guidance into memory-based influence. 
This loss also precipitated a deeper philosophical engagement with beauty as a core value, shaping my artistic ideology. Together, these experiences reveal how ANT elucidates the formation of purpose and identity within a musical life, characterized by both enduring connections and transformative absences.
%######## ESC203: Do not change the next line. This shows your Main body page count.
----Total Pages: \pageref{last_page}
\end{abstract}

\vspace{-2ex}


\section{Introduction}

Understanding the underlying motivations and subconscious beliefs behind our actions can lead to a deeper appreciation for why we pursue certain paths. 
Through an actor-network I have developed, I explore my continued motivation to invest time and energy into musical pursuits. 
Specifically, I examine how an annual musical service called Tenebrae creates order within my network, how the passing of my piano teacher has prompted transformation within this system, and how my philosophical understanding of beauty has guided my decisions.

\begin{figure}[h!]            % h=here, t=top, b=bottom, p=page float
  \centering
  \includegraphics[width=0.9\linewidth]{Figs/Actor_Network.png}
  \caption{My actor network focused on my relationship with music and associated actors such as music teachers, performances, and instruments.}
  \label{fig:actor_network}
\end{figure}

\subsection{Network Stability and Punctualization in Tenebrae}

The Tenebrae tradition represents a remarkably stable ordering within my actor-network, maintained through decades of consistent practice. This stability emerges from how various actors—both human and non-human—have become locked into specific roles through ritualized repetition. The cathedral space itself demands particular acoustic approaches, the musical scores prescribe specific vocal arrangements, and alumni participants expect certain liturgical movements. These elements collectively enforce role consistency across generations of performers.

This locking-in of roles has led to significant punctualization, whereby the complex network of actors (music directors, alumni, architectural space, musical scores, liturgical traditions) becomes black-boxed into a single recognizable entity: "The Tenebrae Service." This punctualization hides the considerable coordination work required to maintain the tradition while ensuring its efficient annual reproduction. The service operates as a unified actor rather than a constantly renegotiated assemblage.

Within this punctualization, certain resistances and preferences become concealed. My own artistic preference for more contemporary musical interpretations, for instance, becomes subsumed beneath the weight of tradition. The network privileges historical authenticity over musical innovation, rendering alternative approaches virtually invisible. This hidden resistance reveals how punctualization can suppress individual preferences in service of network stability.

The endurance of this system derives from several power dynamics. Alumni maintain significant influence as both financial supporters and ritual participants, ensuring the tradition's preservation. The institutional authority of the choir school and cathedral administration provides structural stability, while the non-human actors (the composed scores, the architectural space) exercise power through their resistance to change. The musical works themselves hold particular power as sacred objects created by the school's founder, making alteration appear as transgression rather than innovation.

This network achieves stability not through absence of change but through managed resistance. While musical interpretations have varied across decades, these variations occur within strictly bounded parameters that maintain the tradition's essential character. The power to define these parameters rests with alumni and institutional authorities who share a vested interest in maintaining the tradition's recognizability across generations. Thus, the system maintains stability through strategic punctualization that allows surface-level adaptations while preserving core elements.

\section{Life Without My Piano Teacher}

Without question, the most significant change in my actor-network during the past twelve months has been the permanent severance of my connection with my piano teacher, Ms. Krechkovsky, following her passing. This rupture fundamentally reconfigured the network that had maintained stability for over thirteen years, revealing previously hidden dependencies and activating new actors in unexpected ways.

The change manifested not as a single event but as a cascade of relational adjustments. Most immediately, her absence created a structural void during one of the most musically intensive periods of my life—preparing for my associateship exam. The weekly lessons that had consistently ordered my practice routine, technical development, and musical interpretation disappeared abruptly. This rupture made me acutely aware of how many aspects of my musical practice depended on her guidance: the regular accountability, the immediate feedback on technique, and the shared musical imagination that had developed over more than a decade.

The network responded to this disruption by activating alternative connections. I was placed in the studio of her husband, Mr. Krechkovsky, whose pedagogical approach differed significantly from his wife's. Where Ms. Krechkovsky had emphasized classical precision and structural clarity, Mr. Krechkovsky introduced stronger post-romantic elements, more generous pedal use, and a different conception of musical architecture. This transition revealed that what I had perceived as "piano instruction" was actually a complex assemblage of competing philosophies, techniques, and aesthetic preferences that had been black-boxed under the singular actor "my teacher."

This change also transformed my relationship with the repertoire itself. As I worked through pieces we had studied together, I became increasingly aware of the countless subtle markings, technical suggestions, and interpretive ideas she had embedded in my scores and playing. These material traces—fingerings, phrasing marks, written comments—became actors in their own right, mediating her continued influence despite her physical absence. Her frequent questions ("What landscape do you see? What instrument do you hear?") now emerge from my own memory rather than her voice, shifting her from a present collaborator to an internalized musical conscience.

The implications of this network reconfiguration are profound. Ms. Krechkovsky has transitioned from an active human actor to what we might call a memorial actor—her influence persists through scores, technical approaches, and mental models rather than direct interaction. This transformation has made me more conscious of the temporal dimension of actor-networks: how relationships sediment into habits, how pedagogical approaches outlive their teachers, and how artistic values transmit across generations.

Furthermore, this rupture has heightened my awareness of why I maintain this network at all. The loss of my primary musical guide forced me to articulate my own reasons for pursuing music—not merely as technical mastery or career advancement, but as what I now understand as a commitment to beauty. Her life, fully dedicated to musical beauty, became both model and justification for my continued engagement with this network, even amid its fundamental reconfiguration.

\section{The Power of an Artistic Ideology}

Within my actor-network, the most powerful non-human actor is not a physical object but a conceptual framework: my artistic ideology of beauty. This system of beliefs, heavily influenced by philosopher Roger Scruton's conception of beauty as an ultimate value, exercises considerable power over my musical decisions, practice habits, and ultimately, my identity as a musician.

This ideology maintains power through its capacity to evaluate and justify musical choices. When I approach a new piece or reconsider familiar repertoire, this ideological framework acts as an internalized critic, constantly asking: "Does this interpretation reveal beauty? Does this phrasing pursue elegance? Does this technical choice serve the ultimate value of musical beauty?" The power is maintained because this ideology has become consubstantial with my musical identity—I cannot separate my concept of musicianship from this pursuit of beauty.

Several resistances were overcome to allow this ideology to gain such power. Initially, during my early organ studies with Dr. Ku, I encountered a pedagogical approach that emphasized technical precision and historical authenticity above aesthetic consideration. This created significant cognitive dissonance—while I could execute the music correctly, the performances felt emotionally sterile and failed to satisfy my emerging sense of what music should achieve. The resistance was both technical (adapting to a new instrument) and philosophical (reconciling different aesthetic values).

I ultimately allowed this resistance to be overcome because the alternative—producing music that felt aesthetically unsatisfying—became more painful than the effort of developing a coherent artistic philosophy. The guidance of mentors like Ms. Krechkovsky and Ms. Dunn, who consistently emphasized beauty in our lessons, provided an alternative model that resonated more deeply with my musical instincts. Their human influence gradually sedimented into a non-human ideological framework that now operates independently of their direct input.

This ideology works with me to consolidate several "products": musical performances that strive for beauty, interpretive choices that prioritize elegance over showmanship, and even my broader decision to continue investing time in musical pursuits. It serves as what Latour might call a "mediator"—rather than merely transmitting my musical training, it transforms it according to its own logic.

The power relationship is ultimately symbiotic rather than oppressive. I have granted this ideology power because it provides what Scruton describes as a "reason that needs no further reason"—a justification for musical effort that transcends external validation like exam scores or audience applause. In Maslow's terms, this pursuit of beauty facilitates self-actualization, allowing me to use my musical skills in service of something I perceive as an ultimate value. The ideology maintains its power because I continually renew my commitment to it each time I choose beauty over technical display, elegance over efficiency, or musical truth over popular appeal.

\section{Conclusion}

Music creates lasting connections, as seen in traditions like Tenebrae that unite generations. 
The loss of my teacher revealed how central actors stabilize a network, and how their absence forces reconfiguration. 
This reflection deepened my understanding of beauty as a core value guiding my artistic philosophy. 
Ultimately, these experiences highlight how networks shape purpose and meaning through both presence and loss.


%\bibliographystyle{iclr2022_conference}
%\bibliography{ESC203_Actor_Network}


\label{last_page}

\end{document}