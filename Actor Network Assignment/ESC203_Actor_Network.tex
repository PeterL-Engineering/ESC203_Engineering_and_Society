\documentclass{article} % For LaTex2e
\usepackage{iclr2022_conference,times}
% Optional math commands from https://github.com/goodfeli/dlbook_notation.
\input{math_commands.tex}

%######## ESC203: Uncomment your submission name
%\newcommand{\escname}{ - Project Proposal}
%\newcommand{\escname}{Progress Report}
%\newcommand{\escname}{Final Report}

%######## ESC203: Put your Group Number here
%\newcommand{\gpnumber}{40}

\usepackage{hyperref}
\usepackage{xcolor}
\usepackage[normalem]{ulem}
\usepackage{url}
\usepackage{graphicx}
\usepackage{placeins}
\usepackage{float}
\usepackage{tikz}
\usepackage{multicol}

%######## ESC203: Put your project Title here
\title{An Actor-Network Analysis of Musical Development:\\
Tradition, Pedagogy, and the Pursuit of Beauty}

%######## ESC203: Put your names, student IDs and Emails here
\author{Peter Leong \\
Student\# 1010892955 \\
peter.leong@mail.utoronto.ca \\
\AND
}

% The \author macro works with any number of authors. There are two commands
% used to separate the names and addresses of multiple authors: \And and \AND.
%
% Using \And between authors leaves it to \LaTex{} to determine where to break
% the lines. Using \AND forces a linebreak at that point. So, if \LaTex{}
% puts 3 of 4 authors names on the first line, and the last on the second
% line, try using \AND instead of \And before the third author name.

\newcommand{\fix}{\marginpar{FIx}}
\newcommand{\new}{\marginpar{NEW}}

\iclrfinalcopy 
%######## ESC203: Document starts here
\begin{document}

\maketitle

\vspace{-6ex}

\begin{abstract}
This reflection examines my musical journey through the lens of Actor-Network Theory (ANT), exploring how human and non-human actors form networks that create meaning and motivation. 
I analyze the stability of the longstanding Tenebrae tradition at my choir school, which punctualizes a complex network of alumni, institutions, and rituals into a single enduring actant. 
In stark contrast, the profound loss of my piano teacher demonstrates how the dissolution of a central actor forces network reconfiguration and transforms active guidance into memory-based influence. 
This loss also precipitated a deeper philosophical engagement with beauty as a core value, shaping my artistic ideology. Together, these experiences reveal how ANT elucidates the formation of purpose and identity within a musical life, characterized by both enduring connections and transformative absences.
%######## ESC203: Do not change the next line. This shows your Main body page count.
----Total Pages: \pageref{last_page}
\end{abstract}

\vspace{2ex}

\begin{multicols}{2}

\section{Introduction}

Understanding one's underlying motivations and subconscious beliefs for an action can bring about a deeper appreciation for whic.
Through the assistance of the actor network I have developed, I explore my continued motivation to expend time and energy on my musical pursuits.
More specifically, I discuss how an annual musical service called Tenebrae brings about order to my network, how the passing of my piano teacher has influenced change within this network, and finally, how my understanding of beauty and its philosophy has governed my decisions.

\section{Maintaing Order Through Tradition}

Entering a musical space brings about friendships and connections that can last a lifetime.
Having been at an instituition that fostered these connections for a decade of my upbringing, many of the relationships I have formed have been rooted in musical experiences.
One such tradition that has been a staple of the choir school is Tenebrae.

Tenebrae has been one of if not the longest standing choral tradition at St. Michael's Choir School that brings together generations of choir boys.
Each year during the Wednesday of Holy Week, current students and alumni perform one of the most well-known works of the school's founder, Monsignor John Edward Ronan.
Though during my years as a student at the choir school, Tenebrae was already one of my favourite services as it brought together older students who I met at the school, it has become even more so now that I myself am an alumnus.

\subsection{Staying Connected Through Tenbrae}

One of the greatest beauties of Tenebrae is that it would not be possible without the participation of alumni.
As a result of the consistency that has endured across decades of Tenebrae performances, various actors within the network such as the cathedral basilica and its rectors, the choir school, and the alumni association have become locked into the roles they take on.
Additionally, due to the minimal changes in each actors' roles, the Tenebrae service has largely become punctualized into a single actor whose smaller influencing actants need not be analysed at the microscopic scale.

It is not surprising that over the years, there have been a wide variety of musical styles and interpretations of the Tenebrae responsories (the music sung during the service); however, despite this resistance in the network, its effects have largely been hidden.
I think this is reprepsentative of the shared meaning of Tenebrae to each of its actors, and it reflects that even somewhat significant changes over the course of many years does not change Tenebrae as a punctualized actor.
Furthermore, I believe the system has such enduring stability because the alumni of the choir school have such a strong amount of power and influence over Tenebrae, both as religious service but also as a means to bring as many graduates together.

\section{Life Without My Piano Teacher}

Without question, one of, if not, the biggest changes in my network was the severance of my connection with my piano teacher, Ms. Krechkovsky.
Unlike some other changes that merely resulted in an evolution or a modified dynamic, her passing is a permanent loss which has and will continue to affect me tremendously.
For over 13 years, my weekly piano lesson with her brought order to my network and was often of the highlights of my week.

Apart from her immediate passing, this shocking change increasingly revealed itself throughout various areas of my life.
Most notably, she was no longer there for me to ask for guidance regarding musical inquiries or simply to keep in touch.
I was without a teacher during one of the most intense musical periods of my life thusfar when I prepared for my associateship exam.
I no longer had to visit her studio or make the trip to her home for my lesson.

Though she had passed, I was made aware of new actors who would make a strong impression upon me.
Eventually, I was placed into her husband's studio, Mr. Krechkovsky, who over time modified my personal interpretation of piano repertoire.
He influenced my interpretation to incoroporate stronger post-romantic elements and choices such as the use of pedal.
Additionally, it made me far more aware of the numerous tips or nuances she had bestowed upon me in many pieces of my repertoire.
As I played through them, I would reminisce of my countless hours trying to improve my melodic or chordal voicing.
Or perhaps, asking myself the same questions she would often ask me: "What landscape do you see? What instrument do you hear?"

One of the most persistent implications of this change is the shift of her being a present actor into a memory-based actor.
She no longer influences me directly through conversation or actions, but in some ways interacts with me through the memories she has left behind, such as in her aforementioned nuances or musical ideas.
Also, her passing made me more aware of the reasons for which I spend my time and energy on music, namely the pursuit of beauty.
Though I discuss this further in the subsequent section, I have begun to realize that a life fully committed to the pursuit of beauty such as hers is one that is certainly worth living.

\section{Understanding my Relationship with Philosophy and Beauty}

In the context of Maslow's Hierarchy of needs (needs citation), music and arts in a broader context help fulfill our desire for self-actualization.
Despite all the exams and performances that I have undertaken throughout my years as a musician, ultimately my desire to dedicate my time and energy toward music is motivated by philosophical beliefs of beauty.
Just as an engineer has a worldview that impacts how they approach and apply themselves to projects, I believe that musicians require a artistic ideology that informs their interpretation of repertoire.
Though a musician's artistic ideology need not be informed by one's philosophical biases or beliefs, I think it is natural for both to inform one another as they have for me.
More specifically, I strongly resonate with Roger Scruton's work on describing and understanding beauty. 
He states: "Beauty is an ultimate value–something that we pursue for its own sake, and for the pursuit of which no further reason need be given. Beauty should therefore be compared to truth and goodness, one member of a trio of ultimate values which justify our rational inclinations." (see Scruton Beauty: A very short introduction)

Though I am conscious about these beliefs currently, it has not always been the case, especially during my years in high school.
Several of my most developed musical relationships included in my network such as Ms. Krechkovsky and Ms. Dunn fostered an appreciation for the beauty that can be attained through music.
Further, they encouraged me to evoke landscapes and vivid imagery through the simple beauty ingrained in each musical line and phrase.
Though I was able to develop a subconscious appreciation, I never grasped the underlying beliefs that supported these inclinations till they were challenged by my first organ instructor Dr. Ku.
The organ expanded my palette with its wide range of colours and voices, but my intial instruction did not seem to strengthen my ability to see its potential to achieve beauty.
Regardless of the objective musical validity or conviction of the style he taught me, I felt strongly conflicted and did not feel as though the music I produced satisfied my understanding of beauty.
It took me several years to reflect and realize that the beauty which I sought simply could not exist in this rigid, uncompromising style of interpretation.

Over the subsequent weeks and months, I spent more time developing a purposeful understanding of beauty through philosophy.
Accordingly, through that reflection and applying those conscious beliefs in practice, I feel as though I have granted the power that my artistic ideology possesses over me.
Now, whenever I take to the organ bench or sit in the choir stalls, my ideology and beliefs demand that I achieve this beauty by discovering the innate elegance of each musical line.

\section{Conclusion}

Music creates lasting connections, as seen in traditions like Tenebrae that unite generations. 
The loss of my teacher revealed how central actors stabilize a network, and how their absence forces reconfiguration. 
This reflection deepened my understanding of beauty as a core value guiding my artistic philosophy. 
Ultimately, these experiences highlight how networks shape purpose and meaning through both presence and loss.


%\bibliographystyle{iclr2022_conference}
%\bibliography{ESC203_Actor_Network}

\end{multicols}

\label{last_page}

\end{document}