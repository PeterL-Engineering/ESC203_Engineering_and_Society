\documentclass{article} % For LaTex2e
\usepackage{iclr2022_conference,times}
% Optional math commands from https://github.com/goodfeli/dlbook_notation.
\input{math_commands.tex}

%######## ESC203: Uncomment your submission name
%\newcommand{\escname}{ - Project Proposal}
%\newcommand{\escname}{Progress Report}
%\newcommand{\escname}{Final Report}

%######## ESC203: Put your Group Number here
%\newcommand{\gpnumber}{40}

\usepackage{hyperref}
\usepackage{xcolor}
\usepackage[normalem]{ulem}
\usepackage{url}
\usepackage{graphicx}
\usepackage{placeins}
\usepackage{float}
\usepackage{tikz}
\usepackage{multicol}

%######## ESC203: Put your project Title here
\title{An Actor-Network Analysis of Musical Development:\\
Tradition, Pedagogy, and the Pursuit of Beauty}

%######## ESC203: Put your names, student IDs and Emails here
\author{Peter Leong \\
Student\# 1010892955 \\
peter.leong@mail.utoronto.ca \\
\AND
}

% The \author macro works with any number of authors. There are two commands
% used to separate the names and addresses of multiple authors: \And and \AND.
%
% Using \And between authors leaves it to \LaTex{} to determine where to break
% the lines. Using \AND forces a linebreak at that point. So, if \LaTex{}
% puts 3 of 4 authors names on the first line, and the last on the second
% line, try using \AND instead of \And before the third author name.

\newcommand{\fix}{\marginpar{FIx}}
\newcommand{\new}{\marginpar{NEW}}

\iclrfinalcopy 
%######## ESC203: Document starts here
\begin{document}

\vspace{-4ex}

\maketitle

\vspace{-12ex}

\begin{abstract}
This reflection examines my musical journey through the lens of Actor-Network Theory (ANT). 
I analyze the stability of the longstanding Tenebrae tradition at my choir school, which punctualizes a complex network of alumni, institutions, and rituals into a single enduring actant. 
In stark contrast, the profound loss of my piano teacher demonstrates how the dissolution of a central actor forces network reconfiguration. 
This loss also precipitated a deeper philosophical engagement with beauty as a core value, shaping my artistic ideology. 
Together, these experiences reveal how ANT elucidates the formation of purpose and identity within a musical life.
%######## ESC203: Do not change the next line. This shows your Main body page count.
----Total words: 940
\end{abstract}

\vspace{-2ex}


\section{Introduction}

Understanding the underlying motivations and subconscious beliefs behind our actions can lead to a deeper appreciation for why we pursue certain paths. 
Through an actor-network I have developed, I explore my continued motivation to invest time and energy into musical pursuits. 
Specifically, I examine how an annual musical service called Tenebrae creates order within my network, how the passing of my piano teacher has prompted transformation within this system, and how my philosophical understanding of beauty has guided my decisions.

\begin{figure}[h!]            % h=here, t=top, b=bottom, p=page float
  \centering
  \includegraphics[width=0.9\linewidth]{Figs/Actor_Network.png}
  \caption{My actor network focused on my relationship with music and associated actors such as music teachers, performances, and instruments.}
  \label{fig:actor_network}
\end{figure}

\section{Network Stability and Punctualization in Tenebrae}

The Tenebrae tradition's stability emerges from how various actors have become locked into specific roles through generational tradition.
The musical scores prescribe specific vocal arrangements, and alumni participants expect certain liturgical movements.
This locking-in of roles has led to significant punctualization, whereby the complex network of actors becomes black-boxed into a single recognizable entity: "The Tenebrae Service."
Within this punctualization, certain resistances and preferences become concealed.
The network privileges historical authenticity over musical innovation, resisting the small nuances or changes in interpretation
This hidden resistance reveals how punctualization can suppress individual preferences in service of network stability.

The endurance of this system derives from several power dynamics. 
Alumni maintain significant influence as both financial supporters and ritual participants, ensuring the tradition's preservation. 
The institutional authority of the choir school and cathedral administration provides structural stability, while the non-human actors (the composed scores, the architectural space) exercise power through their resistance to change. 
The musical works themselves hold particular power as sacred objects created by the school's founder, Msgr. John Edward Ronan.

While musical interpretations have varied across decades, these variations occur within strictly bounded parameters that maintain the tradition's essential character. 
The power to define these parameters rests with alumni and institutional authorities who share a vested interest in maintaining the tradition's recognizability across generations. 
Thus, the system maintains stability through strategic punctualization that allows surface-level adaptations while preserving core elements.

\section{Life Without My Piano Teacher}

Without question, the most significant recent change in my network has been the permanent severance of my connection with my piano teacher, Ms. Krechkovsky, following her passing. 
This rupture fundamentally reconfigured the network that had maintained stability for over thirteen years.
Most immediately, her absence created a structural void during one of the most musically intensive periods of my life—preparing for my associateship exam. 
The weekly lessons that had consistently ordered my practice routine, technical development, and musical interpretation disappeared abruptly. 
This rupture made me acutely aware of how many aspects of my musical practice depended on her guidance: the regular accountability, the immediate feedback on technique, and the shared musical imagination that had developed over more than a decade.

My network responded to this disruption by activating alternative connections. 
I was placed in the studio of her husband, Mr. Krechkovsky, whose pedagogical approach differed significantly from his wife's. 
Where Ms. Krechkovsky had emphasized classical precision and structural clarity, Mr. Krechkovsky introduced stronger post-romantic elements, more generous pedal use, and a different conception of musical architecture. 
This transition revealed that what I had perceived as "piano instruction" was actually a complex assemblage of competing philosophies, techniques, and aesthetic preferences that had been black-boxed under a singular actor.

This change also transformed my relationship with the repertoire itself. 
As I worked through pieces we had studied together, I became increasingly aware of the countless subtle markings, technical suggestions, and interpretive ideas she had embedded in my scores and playiing. 
Her frequent questions ("What do you see? What instrument do you hear?") now emerge from my own memory rather than her voice, shifting her from a present collaborator to an internalized musical conscience.
This transformation has made me more conscious of the temporal dimension of actor-networks: how relationships sediment into habits, and how pedagogical approaches outlive their teachers.
Her life, fully dedicated to musical beauty, became both model and justification for my continued engagement with this network.

\section{The Power of an Artistic Ideology}

Within my actor-network, my artistic ideology of beauty asserts itself as the most powerful non-human actor.
This system of beliefs, heavily influenced by philosopher Roger Scruton's conception of beauty as an ultimate value, exercises considerable power over my musical decisions, practice habits, and ultimately, my identity as a musician.
This ideology maintains power by acting as an internalized critic, evaluating and justifying all musical choices. 
Its authority is sustained because it has become consubstantial with my very musical identity.

Several resistances were overcome to allow this ideology to gain such power. 
Initially, during my early organ studies with Dr. Ku, I encountered a pedagogical approach that emphasized technical precision and historical authenticity above aesthetic consideraations.
I ultimately overcame this resistance because producing music that felt aesthetically unsatisfying became more painful than the effort of developing a coherent artistic philosophy. 
The guidance of mentors like Ms. Krechkovsky. Dr. Farahat, and Ms. Dunn, who consistently emphasized beauty in line and phrasing, provided an alternative model that resonated more deeply with my musical instincts. 
Their human influence gradually sedimented into a non-human ideological framework that now operates independently of their direct input.

This ideology helps me consolidate two main 'products': striving for beautiful musical performances and making a conscious decision to invest my time in them.
I have granted this ideology power because it provides what Scruton describes as a "reason that needs no further reason"—a justification for musical effort that transcends any wordly validation.

\section{Conclusion}

Music creates lasting connections, as seen in traditions like Tenebrae that unite generations. 
The loss of my teacher revealed how central actors stabilize a network, and how their absence forces reconfiguration. 
This reflection deepened my understanding of beauty as a core value guiding my artistic philosophy. 
Ultimately, these experiences highlight how networks shape purpose and meaning through both presence and loss.


%\bibliographystyle{iclr2022_conference}
%\bibliography{ESC203_Actor_Network}


\label{last_page}

\end{document}